%%=============================================================================
%% Voorwoord
%%=============================================================================

\chapter*{Voorwoord}
\label{ch:voorwoord}

%% TODO:
%% Het voorwoord is het enige deel van de bachelorproef waar je vanuit je
%% eigen standpunt (``ik-vorm'') mag schrijven. Je kan hier bv. motiveren
%% waarom jij het onderwerp wil bespreken.
%% Vergeet ook niet te bedanken wie je geholpen/gesteund/... heeft

%todo:
Eerst en vooral wil ik de mensen bedanken die mij ondersteund hebben bij dit onderzoek: 
\begin{itemize}
\item Van de Brempt Glenn:	Voor het nalezen en aanpassen van dit onderzoek.
\item Van Vreckem Bert: 	Voor de administratieve en technische hulp bij elke stap van dit onderzoek.
\end{itemize}

Ik heb dit onderwerp gekozen voor een paar redenen, maar de grootste reden is de licentieverandering van nessus. Hierover vertel ik later meer, maar deze zat vroeger in de kali linux distributie. Deze is vervangen door openvas. Sinds ik graag test met verschillende security tools vroeg ik me af of deze 2 scanners van een gelijkwaardig kwaliteit zijn, en wat de belangerijkste verschilpunten hiervoor zijn.  Een andere reden hiervoor is dat beveiliging steeds belangerijk wordt voor bedrijven. Dit heeft te maken met de verspreiding van een aantal grootschalige cryptolockers, en de impact die het op een bedrijf heeft.

Sinds security een constant veranderende wereld is, zijn papers hierover niet lang up to date. Ook zijn er weinig tot geen onderzoeken gevoerd vanuit een neutraal standpunt, hierdoor is dit volgens mij een perfect onderwerp voor mijn onderzoek. Ik kan mijn passies gebruiken terwijl ik een antwoord krijg op mijn persoonlijke vragen, en mensen laten inzien dat security een heel groot deel van elk bedrijf moet zijn. 
