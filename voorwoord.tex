%%=============================================================================
%% Voorwoord
%%=============================================================================

\chapter*{Voorwoord}
\label{ch:voorwoord}

%% TODO:
%% Het voorwoord is het enige deel van de bachelorproef waar je vanuit je
%% eigen standpunt (``ik-vorm'') mag schrijven. Je kan hier bv. motiveren
%% waarom jij het onderwerp wil bespreken.
%% Vergeet ook niet te bedanken wie je geholpen/gesteund/... heeft

%todo:
Eerst en vooral wil ik de mensen bedanken die mij ondersteund hebben voor deze bachelorproef: 


%todo uitbreiden

Ik heb dit onderwerp gekozen voor een paar redenen: 
\begin{itemize}
\item Ik heb me deze vraag ook gesteld, maar nog niet veel neutrale standpunten gehoord.
\item Dit onderzoek kan handig zijn voor bedrijven die hun vulnerability managment willen uitbreiden.
\item Ik kan mijn passies gebruiken voor het schrijven van deze bachelorproef.
\item Nessus was vroeger een deel van kali, maar is vervangen door openvas door licentie redenen, ik wil weten of de kwaliteit van beiden even hoog is.
\end{itemize}

%todo  NOG AANVULLEN OFC; maar wat?!